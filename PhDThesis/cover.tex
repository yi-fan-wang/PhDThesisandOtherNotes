%\documentclass[12pt]{report}
\marginparsep 0pt
\textwidth 15.0 truecm

%\setcounter{page}{0}
%\begin{document}

\setlength{\baselineskip}{0.780cm}


\pagestyle{empty}
\begin{center}
\Large{ \bf Hunting for Primordial Black Holes with Stochastic Gravitational-Wave Background }

\Large{ 用引力波隨機背景探索原初黑洞 }
\end{center}


\vspace{20mm}
\begin{center}
WANG, Yifan

王一帆
\end{center}


\vspace{20mm}
\begin{center}
A Thesis Submitted in Partial Fulfilment \\
of the Requirements for the Degree of \\
 Doctor of Philosophy \\
 in \\
 Physics
\end{center}


\vspace{20mm}
\begin{center}
The Chinese University of Hong Kong\\
October 2019
\end{center}

%=======================================================
\newcommand{\committee}[1]{\def\@committee{#1}}
\newcommand{\thecommittee}{\@committee}
\committee{
	Professor CHU Ming Chung (Chair)\\
	Professor LI Tjonnie Guang Feng (Thesis Supervisor)\\
	Professor WANG Hsiang Hsu (Committee Member)\\
	Professor Huang Qingguo (External Examiner)
}

\newpage
	\thispagestyle{empty}
	
	\begin{center}
		\vspace*{1cm}
		\vfill
		%\Large
		\underline{\bf Thesis Assessment Committee}
		\vskip 1.5cm
		{
			\thecommittee
		}
		\vfill
		\vspace*{8cm}
	\end{center}
	
	
%=======================================================
\newpage
\pagenumbering{roman}
%\setcounter{page}{1}
\pagestyle{myheadings}
%\markright{The Title of My Great Thesis }

\noindent
{\Huge {\bf Abstract}}
\vspace{1.2cm}


\noindent 

The direct detection of gravitational waves has ushered in a new era of observational astronomy.
It has opened a unique window to investigate astrophysics and probe some of the most mysterious matter in the Universe such as dark matter.
This thesis is focused on hunting for primordial black holes, a candidate for dark matter, with a stochastic background of gravitational waves.

In \cref{ch:review1}, we briefly review the basic aspects of the gravitational-wave astronomy, including the Einstein field equation, gravitational wave as a perturbative solution, and the data analysis techniques for gravitational waves from binary compact objects such as black holes or neutron stars.
This chapter lays the foundation for the rest of the thesis.

\cref{ch:SGWBOverview} is dedicated to an overview of stochastic gravitational-wave background.
Besides the gravitational-wave transient signals which are strong enough to be detectable individually, there exist numerous sub-threshold gravitational waves in the Universe whose incoherent superposition forms a stochastic background.
The mathematical characteristics and the cross-correlation data analysis techniques are reviewed.

In \cref{chapter:sgwb-pbh} we discuss an original work which derives the stochastic background energy density spectrum from the coalescence of primordial black holes, which is generated by the direct collapse from the quantum fluctuations of the early Universe.
Using the current upper limit on stochastic background measured by LIGO and Virgo collaboration, we constrain the abundance of primordial black hole with 1 to 100 solar mass.
The stochastic background has given the most stringent upper limit on abundance in this mass range compared with other electromagnetic observations such as microlensing.

\cref{chap:SGWBspace} is another original work investigating the stochastic background from extreme mass-ratio inspiral systems including a primordial black hole and a massive black hole at the galactic center using the space-based gravitational-wave detector.
We model the event rate based on dark matter density profile near the galactic center and the massive black hole population analysis.
The results show that the stochastic gravitational-wave background can be detected if the abundance of primordial black holes with 1 solar mass exceeds a minimum value which is in the range of $\sim10^{-9}$ to $\sim 10^{-6}$ given the modeling uncertainty.
This minimum abundance for detection is also an upper limit for constraint if there will a null searching result.

\cref{ch:outlook} is the outlook for possible future research directions based on the work in this thesis.

\newpage
%\pagenumbering{roman}
%\setcounter{page}{1}
\pagestyle{myheadings}

\noindent {\Huge {\bf 摘要}} \vspace{1.2cm}

引力波的直接探測將觀測天文學帶入了一個全新的時代。
它為研究天體物理和尋找暗物質侯選者打開了一扇新的窗口。
本論文的主題為研究用引力波隨機背景探索原初黑洞。

第一章回顧廣義相對論中的愛因斯坦場方程,作為一個微擾解的引力波解以及致密雙星並合引力波的相關數據分析技術。本部分為論文其余部分的基礎。

第二章回顧引力波隨機背景。
除了可以直接探測到的獨立引力波事件,宇宙中還存在大量的弱引力波信號,它們的非相幹疊加形成隨機背景。
這一章將回顧隨機背景的數學描述及用於尋找隨機背景的交叉關聯數據分析技術。

第三章討論一個原創工作,這一工作計算了原初黑洞並合產生的引力波背景,並利用當前引力波背景觀測上限給出了1至100太陽質量的原初黑洞的豐度的限制。
在此質量區間,相比其他電磁手段例如微引力波透鏡給出的限制,引力波背景給出了最嚴格的限制。

第四章也是一個原創工作,研究由原初黑洞和星系中心的大質量黑洞形成的極端質量比旋進系統產生的引力波背景。
我們基於星系中心暗物質密度輪廓模型和大質量黑洞的數量來建模了這種極端質量比旋進系統的事件率。
結果顯示,只要質量為1太陽質量的原初黑洞的豐度大於某一範圍在$10^{-9}$至$10^{-6}$之間的最小值,該系統產生的隨機背景就有機會被空間引力波探測器探測到。
反過來,如果未來沒有被探測到,我們就可以推斷原初黑洞的豐度上限。

第五章為展望未來可能繼續從事的幾個研究課題。


%=======================================================

\newpage
\noindent {\large {\bf ACKNOWLEDGMENTS}} \vspace{1.2cm}

First of all I would like to express my most sincere gratitude to my supervisor Tjonnie who guided me into the field of gravitational wave astronomy.
It is the luckiest thing for me during my postgraduate period that Tjonnie and I were recruited by the Chinese University of Hong Kong (CUHK) almost at the same time so that I can become his student and do gravitational wave physics.

I thank all the current and former members of the gravitational wave group in CUHK.
I thank Otto Hannuksela who also started postgraduate at the same time as me, your enthusiasm always motivates me. 
I also thank Peter Pang for the paper we collaborated and teaching me so many things, Adrian Chung, I really admire your mathematics expertise, and Patrick, David, Isaac, Kaye, Terrence for making our group a very fruitful place.

I would like to thank the testing general relativity group of the LIGO and Virgo collaboration, I benefit very much from the experience of participating in such a large collaboration.
Thank Qing-Guo Huang and Sai Wang for the academic collaboration.

Above all, I thank my parents for their love and support without any reservation.


\noindent


%=======================================================
